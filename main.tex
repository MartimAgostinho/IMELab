\documentclass[12pt]{article}
% ----------------------------------------------------------------------
% Define external packages, language, margins, fonts, new commands 
% and colors
% ----------------------------------------------------------------------
\usepackage[utf8]{inputenc} % Codification
\usepackage[english]{babel} % Writing idiom

\usepackage{pgfplots}
\pgfplotsset{compat=1.18}
\usepackage[T1]{fontenc}
\usepackage{imakeidx}
\usepackage{circuitikz}

\usepackage[export]{adjustbox} % Align images
\usepackage{amsmath} % Extra commands for math mode
\usepackage{amssymb} % Mathematical symbols
\usepackage{anysize} % Personalize margins
    %\marginsize{2cm}{2cm}{2cm}{2cm} % {left}{right}{above}{below}
\usepackage{appendix} % Appendices
\usepackage{cancel} % Expression cancellation
\usepackage{caption} % Captions
    \captionsetup{labelfont={bf}}
\usepackage{cite} % Citations, like [1 - 3]
\usepackage{color} % Text coloring
\usepackage{fancyhdr} % Head note and footnote
    \pagestyle{fancy}
    \fancyhf{}
    \fancyhead[L]{
        \includegraphics[width=4cm]{NovaFct.png}
    } % Left of Head note
    \fancyhead[R]{\footnotesize Instrumentação e Medidas Elétricas} % Right of Head note
    \fancyfoot[L]{\footnotesize LEEC} % Left of Footnote
    \fancyfoot[R]{\thepage  } % Center of Footnote
    %\fancyfoot[R]{\footnotesize Degree} % Right of Footnote
    \renewcommand{\footrulewidth}{0.4pt} % Footnote rule
\usepackage{float} % Utilization of [H] in figures
\usepackage{graphicx} % Figures in LaTeX
\usepackage[colorlinks = true, plainpages = true, linkcolor = istblue, urlcolor = istblue, citecolor = istblue, anchorcolor = istblue]{hyperref}
\usepackage{indentfirst} % First paragraph
\usepackage[super]{nth} % Superscripts
\usepackage{siunitx} % SI units
\usepackage[list=true, listformat=simple]{subcaption} % Subfigures
\usepackage{titlesec} % Font
    \titleformat{\section}{\Large\bfseries}{\thesection}{1em}{}
    \titleformat{\subsection}{\large\bfseries}{\thesubsection}{1em}{}
    \titleformat{\subsubsection}{\normalsize\bfseries}{\thesubsubsection}{1em}{}
    %\fancyfoot[C]{\thepage}

%code
\usepackage[breakable]{tcolorbox}

\graphicspath{ {images/.} }

% Code highlighting
\usepackage{listings}
\lstset{ 
    backgroundcolor=\color{cellbackground}, % background color for the code block
    basicstyle=\ttfamily,                   % font style
    breakatwhitespace=false,                % automatic breaks only at whitespace
    breaklines=true,                        % automatic line breaking
    captionpos=b,                           % caption position
    commentstyle=\color{gray},              % comment style
    escapeinside={\%*}{*)},                 % if you want to add LaTeX within your code
    keywordstyle=\color{blue},              % keyword style
    stringstyle=\color{dkgreen},            % string style
    frame=single,                           % adds a frame around the code
    rulecolor=\color{cellborder},           % border color
}

% Exact colors from NB
\definecolor{incolor}{HTML}{303F9F}
\definecolor{outcolor}{HTML}{D84315}
\definecolor{cellborder}{HTML}{CFCFCF}
\definecolor{cellbackground}{HTML}{F7F7F7}
\definecolor{commentcolour}{rgb}{0.42, 0.45, 0.51}
\definecolor{string}{rgb}{0.42, 0.45, 0.51}

% Random text (not needed)
\usepackage{lipsum}
\usepackage{duckuments}

% New and re-newcommands
\newcommand{\sen}{\operatorname{\sen}} % Sine function definition
\newcommand{\HRule}{\rule{\linewidth}{0.5mm}} % Specific rule definition
\renewcommand{\appendixpagename}{\LARGE Appendices}

% Colors
\definecolor{istblue}{RGB}{28, 118, 196}
\definecolor{dkgreen}{rgb}{0,0.6,0}

%%%%%%%%%%%%%%%%%%%%%%%%%%%%%%%%%%%%%%%%%%%%%%%%%%%%%%%%%%%%%%%%%%%%%%%%
%                                 Document                             %
%%%%%%%%%%%%%%%%%%%%%%%%%%%%%%%%%%%%%%%%%%%%%%%%%%%%%%%%%%%%%%%%%%%%%%%%
\begin{document}

% ----------------------------------------------------------------------
% Cover
% ----------------------------------------------------------------------
\begin{center}
    \begin{figure}
        \vspace{-1.0cm}
        \includegraphics[scale = 1, left]{NovaFct.png} % Nova logo
    \end{figure}

    \mbox{}\\[2.0cm]
    \textsc{\Huge Licenciatura em Engenharia Eletrotécnica e de Computadores}\\[2.5cm]
    \textsc{\LARGE Instrumentação e Medidas Elétricas}\\[2.0cm]
    \HRule\\[0.4cm]
    {\large \bf { 
        Relatório trabalho laboratorial\\
        Sistema Autónomo de Monitorização do Consumo de\\
        Energia Elétrica de uma Carga Monofásica 
    }[\texttt{PT}]}\\[0.2cm]
    \HRule\\[1.5cm]
\end{center}

\begin{flushleft}
    \textbf{Authors:}
\end{flushleft}

\begin{center}
    \begin{minipage}{0.5\textwidth}
        \begin{flushleft}
            Martim Agostinho (62964)\\
            Diogo Novais (62506)\\
            Filipe Cavalheiro (62894)\\
            Isaac Furtado (62884)
        \end{flushleft}
    \end{minipage}%
    \begin{minipage}{0.5\textwidth}
        \begin{flushright}
            \href{mailto:md.agostinho@campus.fct.unl.pt}{\texttt{md.agostinho@campus.fct.unl.pt}}\\
            \href{mailto:dm.novais@campus.fct.unl.pt}{\texttt{dm.novais@campus.fct.unl.pt}}\\
            \href{mailto:fs.cavalheiro@campus.fct.unl.pt}{\texttt{fs.cavalheiro@campus.fct.unl.pt}}
            \href{mailto:im.furtado@campus.fct.unl.pt}{\texttt{im.furtado@campus.fct.unl.pt}}
        \end{flushright}
    \end{minipage}
\end{center}
    
    % Caixa a dizer o grupo
    %\begin{flushleft}
    %    \large $\boxed{\text{\bf Group} \ \clubsuit}$\\[4.0cm]
    %\end{flushleft}

    \vspace{4cm}

    \begin{center}
        \large \bf 2023/2024 -- 2º Semester
    \end{center}

    \thispagestyle{empty}

    \setcounter{page}{0}

    \newpage

    \newpage

    \tableofcontents % Generates the table of contents

    \newpage

    \listoffigures

    \newpage
    \section{Introdução e contextualização}
    Este relatório detalha o desenvolvimento e a implementação de um 
sistema automático de medida destinado à monitorização do consumo 
de energia elétrica em cargas monofásicas, sem necessidade de parametrização 
por parte do utilizador. O sistema inclui a visualização remota dos
dados através da plataforma ThingSpeak e armazena informações para
análises futuras.

    \section{Solução proposta}
    Neste trabalho é fornecida a saida do transdutor tendo acesso a dois dispositivos sendo os mesmos um transdutor de tensão-tensão
que têm variação de [-2.5;2.5] volt (apartir de \(\pm\) 230v) como é demonstrado na Figura~\ref{fig:transdutor_tensão}, 
e o trandutor de tensão-corrente que têm uma variação á saida de [-1;1] volt (apartir de \(\pm\) 20A), demosntrado na Figura~\ref{fig:transdutor_corrente}.
\vspace*{1cm}
\begin{figure}[H]
    \centering
    \begin{subfigure}[b]{0.5\textwidth}
        \begin{tikzpicture}[scale=0.85]
            \begin{axis}[
            minor tick num=3,
            xtick=\empty,
            ytick={},
            axis y line=left,
            axis x line=middle,
            xlabel={$t$}, ylabel={$V(t)$},
            ymax=3,
            ymin=-3,
            ]
            \addplot[smooth, blue, mark=none, domain=0:10, samples=40] {2.5*cos(deg(x))};
            \end{axis}
        \end{tikzpicture}
        \caption{transdutor de tensão.}
        \label{fig:transdutor_tensão}
    \end{subfigure}%
    \begin{subfigure}[b]{0.5\textwidth}
        \begin{tikzpicture}[scale=0.85]
            \begin{axis}[
            minor tick num=3,
            xtick=\empty,
            ytick={},
            axis y line=left,
            axis x line=middle,
            xlabel={$t$}, ylabel={$I(t)$},
            ymax=3,
            ymin=-3,
            ]
            \addplot[smooth, blue, mark=none, domain=0:10, samples=40] {cos(deg(x))};
            \end{axis}
        \end{tikzpicture}
        \caption{transdutor de corrente.}
        \label{fig:transdutor_corrente}
    \end{subfigure}%
    \caption{Variação dos sinais dos transdutores}
\end{figure}


\noindent O problema dos valores medidos é que o microcontrolador usado, o esp32, não é capas de ler este sinal
mas sim apenas sinais entre [0, 3.3] volt dai ser necessario usar montagem somadora para elevar o sinal para valores 
apenas positivo e reduzir a amplitude na tensão.

\vspace{1cm}

\begin{figure}[H]
  \begin{subfigure}[b]{0.5\textwidth}
    \centering
    \ctikzset{amplifiers/fill=cyan!20, component text=left}
    \begin{circuitikz}
    \draw
    (0, 0) node[op amp, anchor=-] (opamp) {LM741}
    (opamp.-) to[short,-*] ++(-1, 0) coordinate(A)
    (opamp.+) -| ++(-1,-1) node[ground](B){}
    (opamp.out) to [short,*-o] ++(1, 0) coordinate(C) node[above]{$v_o$}
    (A) to[R,l_=$10K\Omega$] ++(-2, 0) -- ++(-1, 0) to ++(0, 0) node[above]{$v_1$}
    (A) |- ++(1,1) coordinate(L1) to[R=$1K\Omega$] ++(2,0) -| (opamp.out)
    (L1) to [R,l_=$2K\Omega$] ++(-4,0) coordinate(D)
    (D) to ++(0, 0) node[above]{$v_2$}
    (L1) to [short,-*] ++(-1, 0)
    ;
    \end{circuitikz}
    \caption{Sinal de corrente}
    \label{fig:circuit_somador_corrente}
  \end{subfigure}%
  \begin{subfigure}[b]{0.5\textwidth}
      \centering
      \ctikzset{amplifiers/fill=cyan!20, component text=left}
      \begin{circuitikz}
      \draw
      (0, 0) node[op amp, anchor=-] (opamp) {LM741}
      (opamp.-) to[short,-*] ++(-1, 0) coordinate(A)
      (opamp.+) -| ++(-1,-1) node[ground](B){}
      (opamp.out) to [short,*-o] ++(1, 0) coordinate(C) node[above]{$v_o$}
      (A) to[R,l_=$10K\Omega$] ++(-2, 0) -- ++(-1, 0) to ++(0, 0) node[above]{$v_1$}
      (A) |- ++(1,1) coordinate(L1) to[R=$1K\Omega$] ++(2,0) -| (opamp.out)
      (L1) to [R,l_=$1K\Omega$] ++(-4,0) coordinate(D)
      (D) to ++(0, 0) node[above]{$v_2$}
      (L1) to [short,-*] ++(-1, 0)
      ;
      \end{circuitikz}
      \caption{Sinal de tensão}
      \label{fig:circuit_somador_tensao}
  \end{subfigure}%
  \caption{Circuito de condicionamento}
\end{figure}

\noindent Neste trabalho queremos registar no ThingSpeak os valores da tensão e 
corrente eficazes, a potência ativa consumida, a potência aparente
consumida, assim como o fator de potência e as frequências da tensão
e corrente.\par
\vspace{1ex}
\noindent Para tal o esp32 passa 1 minuto a recolher amostras, da rede, e após
esse minuto ele processa os dados recolhidos e envia os dados
processados para o ThingSpeak.\par
\vspace{1ex}
\noindent Recolha de dados:\par
\noindent Como já foi referido o esp32 passa 1 minuto a recolher dados,
estes dados são as tensões equivalentes de tensão e correntes
recolhidas pelo transdutor e limitados pelos circuitos acima.
Durante a recolha de dados o esp32 recolhe uma amostra de sinal
a cada 200 us estas amostras são adicionadas à média de todas
as amostras já tiradas para o calculo de alguns valores diferentes.

\begin{figure}[H]
  \begin{subfigure}[b]{0.5\textwidth}
      \centering
      \begin{tikzpicture}[scale=0.85]
          \begin{axis}[
          minor tick num=3,
          xtick=\empty,
          ytick={},
          axis y line=left,
          axis x line=middle,
          xlabel={$x$}, ylabel={$\sin(x)$},
          ymax=4,
          ymin=-1,
          ]
          \addplot[smooth, blue, mark=none, domain=0:10, samples=40] {1.25*cos(deg(x)) + 1.5};
          \end{axis}
      \end{tikzpicture}
      \caption{Tensão}
      \label{fig:esp_tensao}
  \end{subfigure}%
  \begin{subfigure}[b]{0.5\textwidth}
      \centering
      \begin{tikzpicture}[scale=0.85]
          \begin{axis}[
          minor tick num=3,
          xtick=\empty,
          ytick={},
          axis y line=left,
          axis x line=middle,
          xlabel={$x$}, ylabel={$\sin(x)$},
          ymax=4,
          ymin=-1,
          ]
          \addplot[smooth, blue, mark=none, domain=0:10, samples=40] {cos(deg(x)) + 1.5};
          \end{axis}
      \end{tikzpicture}
      \caption{Corrente}
      \label{fig:esp_corrente}
  \end{subfigure}%
  \caption{Sinal recebido}
\end{figure}

Usamos este método para calcular a média da Tensão ao quadrado
\((v(t)^2)\), a média da corrente ao quadrado \((i(t)^2)\) e a média da
potência ativa \((v(t) * i(t))\). 
Também durante este período de recolha
de dados estimamos o valor das frequências de tensão e corrente 
ao detetar quando a tensão ou a corrente passam de um valor 
negativo para um valor positivo, mas não quando passamos de
valor positivo para negativo, e sabendo o número de amostras que
recolhemos entre estes zeros, conseguimos estimar o tempo que
passou entre eles. 

\noindent Processamento de dados: \par
\noindent Depois do minuto de recolha de dados, temos um pequeno intervalo
em que procsaamos os dados recolhidos e enviamos os dados processados
para o ThingSpeak. 
Para obtermos o valor das correntes e tensões eficazes
precisamos apenas de fazer a raiz quadrada das médias da corrente e
tensão quadradas \((\sqrt{i(t)^2})\) e \((\sqrt{v(t)^2})\), respetivamente.\\
O valor da potência aparence é a multiplicação da corrente eficaz com
a tensão eficaz \(S = (\sqrt{(i(t)^2 \cdot v(t)^2)})\). A potência aparente
é a média da tensão a multiplicar pela corrente (fica inalterada depois
da sua medição). O fator de potência é calculado a partir das potências 
aparente e ativa \(cos(\varphi) = (\frac{P}{S})\).
    \section{Resultados e análise}
    \section{Conclusões}
    \section{Bibliografia}
    Slides disponibilizados pelo Professor Rui Miguel Amaral Lopes no Clip da disciplina em "Material Multimédia" da coluna de "Documentação de apoio"
    \section*{Anexo}
    A - codigo do esp32
    \begin{lstlisting}
        #include <WiFi.h>
        #include "secrets.h"
        #include "ThingSpeak.h" // always include thingspeak header file after other header files and custom macros
        
        #define VOLTPIN 34
        #define CURRENTPIN 35
        #define VOLTAGEGAIN 264
        //#define VOLTAGEOFFSET 1.5
        #define SAMPLEPERIOD 200      // unidade - us
        //#define CURRENTOFFSET 1.5
        #define CURRENTGAIN 20
        #define MINPERIOD 10000
        #define SENDPERIOD 60000000
        
        /***********WIFI***********/
        char ssid[] = SECRET_SSID;   // your network SSID (name)
        char pass[] = SECRET_PASS;   // your network password
        int keyIndex = 0;            // your network key Index number (needed only for WEP)
        WiFiClient  client;
        unsigned long myChannelNumber = SECRET_CH_ID;
        const char * myWriteAPIKey = SECRET_WRITE_APIKEY;
        /***********WIFI***********/
        
        bool voltSig    = false; // false = low part of the amplitute true = high
        bool currentSig = false;
        
        // Initialize our values
        float instVolt;
        float instCurrent;
        
        float avgRealPower;
        float apparentPower;
        
        float ief;
        float vef;
        
        float freqc;
        float freqv;
        
        float phi;
        
        
        unsigned long int nSamples = 0;
        unsigned long int auxTimer;
        unsigned long int timeStart;
        
        
        unsigned int nPeriodV;
        unsigned int nPeriodC;
        unsigned long int periodV;
        unsigned long int periodC;
        unsigned long int avgPeriodV;
        unsigned long int avgPeriodC;
        float averageC;
        float averageV;
        unsigned long int periodStartV;
        unsigned long int periodStartC;
        float currentOffset = 1.5;
        float voltageOffset = 1.5;
        
        float ReadVoltage() {
          int x = analogRead(VOLTPIN);
          float voltx = x * 3.3 / 4095;
          return (voltx - voltageOffset) * VOLTAGEGAIN;
        }
        
        float ReadCurrent() {
          int x = analogRead(CURRENTPIN);
          float ic = x * 3.3 / 4095;
          return (ic - currentOffset) * CURRENTGAIN;
        }
        
        void getfreqv() {
          freqv = 1000000.0 / avgPeriodV;
        }
        
        void getfreqc() {
          freqc = 1000000.0 / avgPeriodC;
        }
        
        void getvef() {
          vef = sqrt( vef );
        }
        
        void getief() {
          ief = sqrt( ief );
        }
        
        void getRealpower() {
          avgRealPower = (avgRealPower * nSamples + (instCurrent * instVolt)) / (nSamples + 1);
        }
        
        void getapparentpower() {
          apparentPower = vef * ief;
        }
        
        void getphi() {
          phi = avgRealPower / apparentPower;
        }
        
        void setup() {
          Serial.begin(115200);  //Initialize serial
          while (!Serial) {
            ; // wait for serial port to connect. Needed for Leonardo natinstVolte USB port only
          }
        
          /***********WIFI***********/
          WiFi.mode(WIFI_STA);
          ThingSpeak.begin(client);  // Initialize ThingSpeak
          /***********WIFI***********/
        
          voltSig    = true; //start allways in the beggining of a up wave
          currentSig = true;
        
          auxTimer = micros();
          timeStart = micros();
          periodC = 0;
          periodV = 0;
        }
        
        void loop() {
          Serial.println("Hello");
        
          //Sync to voltage period
          while (instVolt > 0.05) {
            instVolt = ReadVoltage();
          };
          while (instVolt < -0.05) {
            instVolt = ReadVoltage();
          };
        
          auxTimer      = micros(); //Used to count 60 seconds
          nSamples      = 0;
          nPeriodV      = 0;
          nPeriodC      = 0;
          ief           = 0;
          vef           = 0;
          avgPeriodV    = 0;
          avgPeriodC    = 0;
          periodStartV  = auxTimer;
          periodStartC  = auxTimer;
        
          while (micros() - auxTimer < SENDPERIOD) {
            timeStart   = micros();
            instVolt    = ReadVoltage();
            instCurrent = ReadCurrent();
        
            vef = (vef * nSamples + pow(instVolt, 2)) / (nSamples + 1);
            ief = (ief * nSamples + pow(instCurrent, 2)) / (nSamples + 1);
        
            if ((instVolt > 0.05) && !voltSig) { // means it got positinstVolte
        
              periodV  = micros() - periodStartV;
              //if ( periodv > MINPERIOD  ) { Serve para existir uma frequencia minima
        
              periodStartV = periodStartV + periodV;
              avgPeriodV = (avgPeriodV * nPeriodV + periodV) / (nPeriodV + 1);
              nPeriodV++;
              voltSig = true;
              //}
            }
            if ((instVolt < -0.05) && voltSig)
              voltSig = false;
        
            //Current
            if ((instCurrent > 0.01) && !currentSig) { // means it got positinstVolte
        
              periodC  = micros() - periodStartC;
              //if ( periodC > MINPERIOD  ) { Serve para existir uma frequencia minima
        
              periodStartC = periodStartC + periodC;
              avgPeriodC = (avgPeriodC * nPeriodC + periodC) / (nPeriodC + 1);
              nPeriodC++;
              currentSig = true;
              //}
            }
            if ((instCurrent < -0.01) && currentSig)
              currentSig = false;
        
            getRealpower(); // add the read value to the avg Real Power value
        
            while ( micros() - timeStart < SAMPLEPERIOD ) {}
            nSamples++;
          }
          
          getfreqv();
          getfreqc();
          getief();
          getvef();
          getapparentpower();
          getphi();
        
          /***********WIFI***********/
          if (WiFi.status() != WL_CONNECTED) {
            Serial.print("Attempting to connect to SSID: ");
            Serial.println(SECRET_SSID);
            while (WiFi.status() != WL_CONNECTED) {
              WiFi.begin(ssid, pass);  // Connect to WPA/WPA2 network. Change this line if using open or WEP network
              Serial.print(".");
              delay(5000);
            }
            Serial.println("\nConnected.");
          }
        
          // write to the ThingSpeak channel
          int x = ThingSpeak.writeFields(myChannelNumber, myWriteAPIKey);
          if (x == 200) {
            Serial.println("Channel update successful.");
          }
          else {
            Serial.println("Problem updating channel. HTTP error code " + String(x));
          }
          ThingSpeak.setField(1, vef);
          ThingSpeak.setField(2, freqv);
          ThingSpeak.setField(3, ief);
          ThingSpeak.setField(4, freqc);
          ThingSpeak.setField(5, avgRealPower);
          ThingSpeak.setField(6, apparentPower);
          ThingSpeak.setField(7, phi);
          /***********WIFI***********/
        }
        
    \end{lstlisting}
\end{document}